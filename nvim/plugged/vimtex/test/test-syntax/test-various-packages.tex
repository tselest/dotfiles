\documentclass{beamer}
\usepackage{subfile}
\usepackage{includegraphics}
\usepackage{pdfpages}
\usepackage{hyperref}
\usepackage{testx} " Package does not exist - test try catch block
\usepackage{biblatex}
\usepackage{csquotes}
\usepackage{varioref}
\usepackage{dot2texi}
\usepackage{luacode}
\usepackage{gnuplottex}

\begin{document}

\section{General}
\todostuff \todocommand \todo{content}

This text is \emph{emphasized}, \textit{italized}, \textbf{bolded}.

\def\A{A}
\def\B{\V{B}}

\newenvironment{example}[1][]{Start #1}{Stop}
\renewenvironment{example}[1][]{Start #1}{Stop}
\newenvironment{nestedexample}[1][]{%
  \renewenvironment{nested}[1][0]{%
    Outerarg #1}{%
    Innerarg ##1
  }%
  \begin{nested}{title}}{%
  \end{nested}%
}

\newcommand{\testA}[1]{Test #1}
\renewcommand{\testA}[1]{Test #1}
\newcommand{\mycommand}[1]{%
  \renewcommand{\myothercommand}[1]{%
    \dosomethingwithouterarg{#1}%
    \dosomethingwithinnerarg{##1}%
  }%
}

\section{Documentclass: beamer}
\begin{frame}[asd]
  \includegraphics[height=0.5062\linewidth]{test.png}
  \includegraphics<2>[height=0.5062\linewidth]{test.png}

  \only{asd}
  \only<2>{asd}

  \item asdasd
  \item<1-3> asdasd
\end{frame}

\section{Package: minor stuff}
% subfile
\subfile{test-syntax.tex}

% pdfpages
\includepdf{test.pdf}

\section{Package: biblatex and csquotes}
\cite{}
\citet*{}
\citep{bibtexkey1}
\citep[e.g.][]{bibtexkey2}
\citealt{}
\citealt*{}
\citealp{}
\citealp*{}
\citenum{}
\citetext{}
\citeauthor{asd}
\citeauthor*{}
\citeyear{}
\citeyearpar{}
\bibentry{}

\section{Package: varioref}
\Vref{sec_one} does what you need. Still \vref{sec_one} works.

\section{Package: dot2texi}
\begin{dot2tex}
  graph graphname {
    a -- b;
    b -- c;
    b -- d;
    d -- a;
  }
\end{dot2tex}

\section{Package: luacode}
\directlua{
  if pdf.getminorversion() \string~= 7 then
    print "pfd version 1.7"
  end
}

\begin{luacode}
  if pdf.getminorversion() \string~= 7 then
    print "pfd version 1.7"
  end
\end{luacode}

\section{Package: gnuplottex}
\begin{gnuplot}[terminal=..., terminaloptions=...]
    plot file using 1:2 with lines
\end{gnuplot}

\end{document}
